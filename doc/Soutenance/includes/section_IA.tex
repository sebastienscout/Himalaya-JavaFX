\section{L'intelligence artificielle}

\begin{frame}
	\frametitle{L'intelligence artificielle}
	\framesubtitle{Présentation de l'algorithme}
	
	\begin{block}<1->{Les paramètres}
		\begin{itemize}
			\item Population des parents (mu) = 20
			\item Population des enfants (lambda) = 500
			\item Taille du tournoi pour la sélection = 2
			\item Taux de crossing-over = 0.8
			\item Taux de mutation = 1.0
			\item Nombre de générations maximum = 25
		\end{itemize}
	\end{block}	
	
	\begin{exampleblock}<2>{Individu d'une population}
		\begin{itemize}
			\item Suite des 9 meilleurs actions
			\item L'IA réalise les 6 premières actions
		\end{itemize}
	\end{exampleblock}	
\end{frame}

\begin{frame}
	\frametitle{L'intelligence artificielle}
	\framesubtitle{Calcul de la fitness}
	
	\begin{block}<1->{Calcul de la fitness}
		\begin{itemize}
			\item Copie du plateau de jeu, et de tous ses composants
			\item Pas de copie des joueurs
			\item Réalisation d'une simulation du tour pour calculer la fitness
			\item Ne pas trop insister sur les 3 dernières actions
		\end{itemize}
	\end{block}

	\begin{block}<2>{Évaluation des coefficients}
		\begin{itemize}
			\item Coefficients égaux au départ
			\item Adaptation au score du joueur
		\end{itemize}
	\end{block}
\end{frame}

\begin{frame}
	\frametitle{L'intelligence artificielle}
	\framesubtitle{Calcul de la fitness}
	\lstinputlisting[language=Java, caption= {Pseudo-code: Calcul de la fitness}]{code/algo_fitness.java}
\end{frame}

\begin{frame}
	\frametitle{L'intelligence artificielle}
	\framesubtitle{Calcul de position et optimisation des régions}
	
	\begin{block}<1->{Position initiale}
		\begin{itemize}
			\item Run de l'algorithme pour chaque village disponible
			\item Choisir la meilleure fitness
			\item Ajouter le village à la liste des indisponibles
		\end{itemize}
	\end{block}	
	
	\begin{exampleblock}<2>{Région optimale des délégations}
		\begin{itemize}
			\item Liste des régions possibles
			\item Calcul de rentabilité : prise de région, vide ou occupée
			\item Prise en compte des délégations présentes du joueur
		\end{itemize}
	\end{exampleblock}	
\end{frame}