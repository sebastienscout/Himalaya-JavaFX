\section{Conclusion}

À partir de ces éléments, nous pensons avoir réussi à créer une application fonctionnelle pouvant être utilisée pour jouer à plusieurs et contre des IAs. Nous avons suivi au maximum ce qui a été établi dans le cahier des charges en début de projet, de manière à fournir une application fonctionnelle et utilisable. 
Concernant la méthode de travail employée durant le projet, nous nous sommes fixé des objectifs de développement pour chaque semaine, ce qui a permis de savoir sur quoi travailler et de suivre l'avancée du projet. Nous avons souvent travaillé en pair-programming, par exemple sur la gestion de la fitness pour l'algorithme génétique, ou encore sur la création des interfaces.
Le travail en pair-programming permet d'augmenter la rapidité de travail et d'améliorer la réflexion pour trouver une solution à un problème donné.
Ce projet nous a également offert la possibilité de mettre en place des méthodes et algorithmes vus en cours, dans un projet réel, ce qui nous a permis de vérifier que nous avions bien acquis ses compétences.